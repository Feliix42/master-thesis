% !TEX root = ../thesis.tex
%
\chapter{Background and Motivation}%
\label{sec:background}

In this chapter, we are going to introduce the term \emph{Irregular Application}, the concept of \emph{Amorphous Data Parallelism} as well as the parallelism frameworks \emph{Software Transactional Memory} and \emph{Ohua} as a foundation for the following chapters.
We will also briefly discuss the difficulties that arise when reasining about said application types and motivate, why further research into this topic could prove valuable.

\section{Software Transactional Memory}

With the invention of multi-threaded programming, a need for synchronization primitives arose to allow safe parallelization of data-processing applications.
Usually, locks have been used by developers to guard access to pieces of shared data.

Lock-based programming however, has the fundamental drawback of being blocking, meaning that developers have to employ great care when using them in order to not accidentally produce a deadlock (i.e., a state where a number of threads may never again progress) by acquiring locks in the wrong order.
This means a lack of composability, as combining several small lock-based modules into a larger program would require the imposition of some sort of ordering on the locks, which frequently leads to problems.
Even when written correctly, lock-based programs tend to quickly become hard to read and maintain due to the many rules that need to be enforced by the programmer herself without any external checks.

% TODO: figure + ex. code
%\begin{figure}[b]
    %\includegraphics[]{}
    %\caption{}
    %\label{fig:background:stm}
%\end{figure}

Hence, Shavit et al. \cite{shavit1997software} proposed a concept called \emph{Software Transactional Memory}.
This new approach to synchronization aimed to provide lock-free parallelism abstractions that allow multiple threads access to a shared variable without any form of blocking.
Former critical sections are now regarded as \emph{transactions}: They are either executed in their entirety with all their changes taking effect or are not executed at all, just like transactions in databases.
STM's operating principle is outlined in Figure \ref{fig:background:stm} along with a code example.
Every read and write operation to or from a piece of shared data is conducted inside a transaction block and gets initially saved to a local transaction log.
When the transaction block comes to an end, the changes made to individual shared data sections are committed.
Therefore, the changes in the log are applied to the original shared values.
In case another transaction managed to commit in the meantime to a value our transaction also touched, the conflict is detected and instead of committing the changes the transaction is aborted and restarted.
This is normally done until the transaction committed successfully.

In our example in Fig. \ref{fig:background:stm}, the transactions 1 and 2 both read the same value and modify it.
But since the first transaction was able to commit earlier, the second transaction now stands in conflict and has to be aborted and scheduled for re-execution.

As the example already shows, a fundamental benefit of the (Software) Transactional Memory model is its serializability \cite{swalens2016transactional}:
All transactions seem to execute serially, since the steps of one transaction never appear to be interleaved with the steps of another transaction.
Therefore, the results of an execution must be equal to the result of a serial execution.
This has been formalized by Swalens et al. \cite{swalens2016transactional} in their proposed operational semantics for a language with transactions.

Overall, STM takes an optimistic approach to parallelism, because it simply executed numerous computations without any heuristics or reasoning in parallel, hoping for as few conflicts as possible.
The result of this is an underlying non-determinism that ensues everytime transactions are employed in a multi-threaded environment.
Problems in this strategy become apparent when applied to high-contention scenarios.
Since STM may work in parallel over the same data structure or memory region, high contention always leads to a significantly increased number of retries for individual executions, which in turn leads to drastically reduced performance.
Further shortcomings of this concept have been discussed in detail by Caşcaval et al. \cite{cascaval2008software}.
One the one hand, exception handling becomes impossible to do inside of a transaction without breaking its semantics\todo{Was that what they meant?}.
On the other hand, I/O operations cannot be transactionalized, as well as anything else that producs side effects outside of te transactions scope as these effects may not be rolled back on error.
Additionally, they reported large overheads of STM applications for smaller worksets as well as no debuggability, since the non-determinism makes certain situations nearly irreproducable.

\section{Irregular Applications}

\subsection{Amorphous Data Parallelism}




\section{Ohua}



% - present both Ohua and STM
%   - explain paradigms in detail (most people here don't know what stm is and how it works!)
%   - detail benefits and shortcomings of both
%   - Ohua: talk about the compiler, its stages and where optimizations hook in as this will be required information later on.
%   - STM: Detail that non-determinism is part of the execution model -> speculative parallelism
%       - make sure to thoroughly explain the concept of a transaction
%   - Ohua: detail determinism of the model if appropriate, but remember that this will also be discussed later on
% - explain Irregular Applications: They are different from the parallelizability problems we usually face as the input dictates the parallelism at runtime and the computing times!
%   - use: \label{sec:background:irregular}
