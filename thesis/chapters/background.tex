% !TEX root = ../thesis.tex
%
\chapter{Background and Motivation}%
\label{sec:background}

In this chapter, we are going to introduce the term \emph{Irregular Application}, the concept of \emph{Amorphous Data Parallelism} as well as the parallelism frameworks \emph{Software Transactional Memory} and \emph{Ohua} as a foundation for the following chapters.

\section{Irregular Applications}

\subsection{Amorphous Data Parallelism}


\section{Software Transactional Memory}


\section{Ohua}



% - present both Ohua and STM
%   - explain paradigms in detail (most people here don't know what stm is and how it works!)
%   - detail benefits and shortcomings of both
%   - Ohua: talk about the compiler, its stages and where optimizations hook in as this will be required information later on.
%   - STM: Detail that non-determinism is part of the execution model -> speculative parallelism
%       - make sure to thoroughly explain the concept of a transaction
%   - Ohua: detail determinism of the model if appropriate, but remember that this will also be discussed later on
% - explain Irregular Applications: They are different from the parallelizability problems we usually face as the input dictates the parallelism at runtime and the computing times!
%   - use: \label{sec:background:irregular}
