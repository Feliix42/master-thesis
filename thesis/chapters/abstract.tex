% !TEX root = ../thesis.tex
%
\chapter*{Abstract}
\label{sec:abstract}

The efficient parallelization of shared state applications has been an ongoing research topic ever since the advent of multi-core processors.
Due to the properties some of these programs exhibit, such as amorphous data parallelism and the use of pointer-based data structures, conventional concurrency control mechanisms like locking fail to uncover any meaningful parallelism.
Speculative parallelism approaches have been developed as a result, the most prominent representative being \emph{Software Transactional Memory}.
But these frameworks have their own set of drawbacks like the lack of scalability and high overheads.
It is unclear, whether approaches like implicit parallelism could be a feasible and more performant alternative to use for developing these applications.

This thesis tries to find out, whether \emph{Ohua}, a framework for exploiting implicit parallelism, could be used for writing shared state programs and whether it could be an alternative to the by now aged STM.
In order to do this, we tested in a preliminary study Ohua's usability in shared state environments. % as this had not been tested yet.
We then proposed a set of transformations for Ohua to run at compile-time to automatically recognize and exploit parallelism from such applications.
To compare Ohua then against STM, we tested both in a series of benchmarks that ressemble real-world applications.

The evaluation shows that Ohua is not able to exploit parallelism in shared state applications to the same degree as STM does.
Nonetheless, Ohua shows results that are widely on par with Software Transactional Memory and performs especially good in applications with amorphous data parallelism, validating our proposed transformations.
