% !TEX root = ../thesis.tex
%
\chapter{Experimental Setup}
\label{sec:experiments}

In order to compare the performance of Software Transactional Memory against Ohua, we employ a set of benchmarks originally proposed by Minh et al. \cite{minh2008stamp}.
In this chapter, we will categorize the benchmarks introduced by the authors and present a representative selection of applications which we will use to compare our performance against STM.
Additionally, we will discuss the values measured during execution of the benchmarks and their relevance for the evaluation.

\section{Benchmark Choice}

% TODO: Bring in something about irregular applications/amorphous data parallelism?

To provide a fair and comprehensive evaluation for Ohua and the transformations introduced in chapter \ref{sec:transformations}, as well as the STM reference implementation in Rust, we chose the \emph{Stanford Transactional Applications for Multi-Processing} suite.
This benchmark suite was designed by Minh et al. to provide a wide variety of applications and use cases to test Transactional Memory applications against.
With their work, the authors attempt to cover applications from different domains with varying use cases for TM applications.
Therefore, applications exhibit varying characteristics regarding the time they spend in transactions, frequency of transaction usage and contention.
We use a subset of the benchmarks the authors presented to look into Ohua's performance and identify scenarios in which we may under- or outperform STM.

% tables and conclusions from my earlier notes
% explain why i chose bench xy
% explain benches


% - Part 1: Benchmark Choice
%   - which benchmarks did I choose for my thesis, and why? outline the parameters and reasons for the decision
%   - discuss parameters we used to run the benchmarks
%   - Give short overview over the used benchmarks

% \section{Measured Values}
% - Part 2: relevant data points: What did we measure? What does it say about performance?
%   - execution time
%   - CPU time (as substitute for power consumption)
