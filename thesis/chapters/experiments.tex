% !TEX root = ../thesis.tex
%
\chapter{Experimental Setup}%
\label{sec:experiments}

In order to compare the performance of Software Transactional Memory against Ohua, we employ a set of benchmarks originally proposed by Minh et al.~\cite{minh2008stamp}.
In this chapter, we will categorize the benchmarks introduced by the authors and present a representative selection of applications which we will use to compare Ohua's performance against STM.
Additionally, we will discuss the values measured during execution of the benchmarks and their relevance for the evaluation.

\section{Benchmark Choice}
\label{sec:experiments:choice}

\todo{Maybe mention irregular applications/amorphous data parallelism?}
After presenting our transformations for Ohua in chapter~\ref{sec:transformations}, we now wanted to compare its performance against STM in order to evaluate if Ohua could indeed be used as a suitable replacement for developing parallel irregular applications.
In order to provide a comprehensive comparison, we chose to use the \emph{Stanford Transactional Applications for Multi-Processing} suite~\cite{minh2008stamp}.
Introduced by Minh et al., it was designed as a set of benchmarks for testing software transactional memory frameworks.
The authors included 8 applications from different application areas in their suite.
These are supposed to resemble the diverse landscape of parallelism in applications developers might face.
In particular, the STAMP suite contains examples from different application domains and varying use cases for transactional memory such as high-contention and low-contention scenarios.

\begin{table}
    \centering
    \begin{tabular}{|l|l|l|}
        \hline
        \textbf{Application} & \textbf{Instructions per Transaction} \emph{(mean)} & \textbf{Time spent in transactions}\\\hline\hline
        labyrinth & 219,571 & 100\%\\\hline
        bayes & 60,584 & 83\%\\\hline
        yada & 9,795 & 100\%\\\hline
        vacation & 3,223 & 86\%\\\hline
        genome & 1,717 & 97\%\\\hline
        intruder & 330 & 33\%\\\hline
        kmeans & 117 & 7\%\\\hline
        ssca2 & 50 & 17\%\\\hline
    \end{tabular}
    \caption{A basic characterization of STAMP applications, comparing the mean number of instructions per transaction and the overall percentage of time the application spends in transactions. These numbers stem from a C implementation and have been adapted from Minh et al.~\cite{minh2008stamp}}
    \label{tab:experiments:overview}
\end{table}

The tables~\ref{tab:experiments:overview} and~\ref{tab:experiments:categorization} give a basic characterization of the benchmarks in terms of their usage of transactions.
As can be seen in table~\ref{tab:experiments:overview}, length of individual transactions varies greatly per benchmark, as does the overall time that is spent by the benchmark executing transactions.
Even though the numbers in the table have been adapted from Minh et al.\ and represent values measured for their C-based implementation, they still outline the general behavior of an STM-based algorithm.
Some applications suffer so badly from the irregular properties that have been outlined in chapter~\ref{sec:background:irregular} that exploiting their parallelism requires them to spend more than 80\% of their overall execution time in transactions.
This is for example the case in the \emph{labyrinth} benchmark, where fields of a dense 3-dimensional matrix have to be continuously updated.
Other applications such as \emph{kmeans} or \emph{ssca2} have relatively short transactions since their data parallelism is easier to exploit or they generally do not feature as much opportunities for parallelism as other applications.

\begin{table}
    \centering
    \begin{tabular}{|l|l|l|l|l|}
        \hline
        \textbf{Application} & \textbf{tx length} & \textbf{r/w set} & \textbf{tx time} & \textbf{Contention}\\\hline\hline
        labyrinth & long & large & high & high\\\hline
        bayes & long & large & high & high\\\hline
        yada & long & large & high & medium\\\hline
        vacation & medium & medium & high & low/medium\\\hline
        genome & medium & medium & high & low\\\hline
        intruder & short & medium & medium & high\\\hline
        kmeans & short & small & low & low\\\hline
        ssca2 & short & small & low & low\\\hline
    \end{tabular}
    \caption{A qualitative summary of each STAMP application's runtime transactional characteristics. The length of a transaction is determined by the number of instructions it encompasses. The characteristics are ranked relative to the other applications in the suite. Adapted from Minh et al.~\cite{minh2008stamp}}
    \label{tab:experiments:categorization}
\end{table}

Another relevant and the perhaps most limiting factor for programs relying on optimistic parallelism principles such as STM is contention.
This characteristic is visualized in table~\ref{tab:experiments:categorization}\todo{talk about r/w set or leave out} along with other properties.
When facing high contention scenarios, STM implementations are usually unable to achieve near-linear speedups as it is the case for other benchmarks.
The contention is a byproduct of the frequent writing accesses to the shared data structure that inevitably lead to frequent conflicting write accesses which require a rollback of all affected threads except for the one that committed its changes first.
Frequent rollbacks and the accompanying recomputations are the root cause for this behavior.

Based on the analysis provided by Minh et al., we selected a representative range of benchmarks for our comparison between Ohua and STM.
To do so, we chose benchmarks with varying transaction length, varying frequency of transaction use as well as different levels of contention.
The next sections will briefly outline the details of the chosen applications and explain, how they were implemented.


% THEN elaborate further about the technical details that hold true for all benchmarks
Since the authors only provided a C-based reference implementation for their applications, we had to re-implement them in Rust to rule out language-specific performance changes when comparing to Ohua's Rust implementation.
Upon inspecting the original source code, it turned out that the authors adapted the code in order to improve the performance of STM in some benchmarks.
For instance, they provided their own implementation of a \rust{HashMap}\footnote{A dynamic key-value store allowing fast lookups by hashing the keys and organizing them in different buckets.} that offered the use of transactions on a per-bucket basis, effectively exposing fine-granular parallelization opportunities that normal HashMaps cannot provide.
In our Rust implementation, we refrained from using such optimizations for either STM or Ohua.
Instead, we implemented both algorithms from scratch based on the descriptions provided by the authors, literature they cited and the code they supplied.
We did so in an idiomatic way, applying both concepts to the problems while only using the tools both frameworks provide natively.
As a result, the categorizations from the tables~\ref{tab:experiments:overview} and~\ref{tab:experiments:categorization} might no longer hold true as contention may have increased by employing more coarse-grained tools in our implementations.

For each example application, the authors also provided at least three different parameter classes or inputs to model small, medium and large workloads which we largely adopted for our benchmarks.


%TODO:
%- explain why i chose bench xy
%- explain benches
%  - detail for labyrinth (etc) why they are so hard to parallelise

\subsection{Labyrinth Path Mapping}
\label{sec:experiments:labyrinth}
%The \emph{labyrinth} benchmark implements a variation of Lee's algorithm \cite{lee1961algorithm} which was conceived to solve path-connection problems that often arise e.g., when searching for an optimal route or when constructing wiring diagrams in electrical engineering.
%Watson et al.~\cite{watson2007study} were the first to implement the algorithm using transactional memory techniques to measure the performance on such systems.
%Based of their algorithm description, we implemented the benchmark, as it has been described in chapter~\ref{sec:preliminary}\todo{verify correctness of this link}.
%\todo{complete this: Ohua-Impl -> was ist parallel?, original perf \& expected perf}

\subsection{Intruder Detection}
\label{sec:experiments:intruder}

\subsection{Genome Sequencing}
\label{sec:experiments:genome}
%This benchmark implements a \emph{\enquote{whole-genome shotgun sequencing}} algorithm as outlined by Pop et al.~\cite{pop2002genome} in their work.
%Its goal is to sequence a (fictional) genome, i.e.\ to reassemble a nucleotide sequence from a set of snippets which is something frequently done in genetics.

%The first step in the algorithm is to deduplicate the DNA segments that have been provided as inputs, since there are usually many duplications.
%The second phase is then concerned with finding neighboring segments in the remaining pool of DNA parts by utilizing overlap matching.
%By reducing the overlap size each iteration, the best possible fit is chosen for each neighbor search.\todo{Visualize this!}
%After this phase has finished, all but two segments have a predecessor and successor assigned.
%Starting from the one element in the set with no predecessor, the chain of nucleotides can be rebuilt by simply following the links.

%In STM, the deduplication phase as well as the matching phase are implemented in parallel.

% As was the case in the \emph{intruder} benchmark, Minh et al.\ used their own, transaction-ready 
% Effectively, phase two of the algorithm boils down to the construction of a linked list from a set of elements.

\subsection{K-means Clustering}
\label{sec:experiments:kmeans}






% - Part 1: Benchmark Choice
%   - which benchmarks did I choose for my thesis, and why? outline the parameters and reasons for the decision
%   - discuss parameters we used to run the benchmarks
%   - Give short overview over the used benchmarks

% \section{Measured Values}
% \label{sec:experiments:measurements}
% - Part 2: relevant data points: What did we measure? What does it say about performance?
%   - execution time
%   - CPU time (as substitute for power consumption)
